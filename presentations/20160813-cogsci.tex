\documentclass{beamer} %[handout]
%\usepackage{listings}
%\usepackage{textcomp}
%\usepackage{units}
%\usepackage{tikz}

%\usetikzlibrary[shapes,snakes, positioning]

\usepackage{color}
\usepackage{picture}
\usepackage{graphicx}

%%\usetheme{Malmoe}
\usetheme{metropolis} %%%
\metroset{block=fill}
\setbeamercolor{alerted text}{ fg=red!65!black}
%\setbeamertemplate{itemize items}[default]
%\setbeamertemplate{enumerate items}[default]
\setbeamercovered{transparent=50}
%\setbeamercovered{still covered={\opaqueness<1->{0}},again covered={\opaqueness<1->{35}}}
\setbeamertemplate{footline}{\hfill \insertframenumber/\inserttotalframenumber ~~~}

%\setbeamertemplate{bibliography entry title}{}
%\setbeamertemplate{bibliography entry location}{}
%\setbeamertemplate{bibliography entry note}{}


%\usepackage[natbib=true, bibstyle=authoryear, citestyle=authoryear-comp, backend=bibtex]{biblatex}
%\usepackage{natbib}
\usepackage{bibentry}
\nobibliography*
%\bibpunct[:]{(}{)}{,}{a}{}{,}
%\usepackage[ngerman]{babel}
\usepackage[T1]{fontenc}
    \usepackage[latin1]{inputenc}
    \usepackage{lmodern} 


\usepackage{amsmath}
\usepackage{amsthm}
\usepackage{amssymb}
\usepackage{hyperref}
\usepackage[T1]{tipa}
\usepackage{gb4e}
\usepackage{tikz}
\usepackage{marvosym}

\usetikzlibrary{trees}
\usetikzlibrary{matrix}
\newcommand{\citeposs}[2][]{\citeauthor{#2}'s (\citeyear[#1]{#2})}
\newcommand{\valueof}[1]{$\llbracket${\em #1}$\rrbracket$} 
%\newcommand{\valueof}[1]{$[[${\em #1}$]]$} 
%\newcommand{\mvalueof}[1]{$[[$ #1 $]]$} 
\newcommand{\mvalueof}[1]{\llbracket#1\rrbracket}
\newcommand{\vFill}{\vskip0pt plus 1filll}
\newcommand{\hFill}{\hskip0pt plus 1filll}
\newcommand{\bibj}[1]{$\cdot$ \bibentry{#1}\\}
%	\newcommand{\argmax}[1]{\underset{#1}{\operatorname{arg}\,\operatorname{max}}\;}
\definecolor{whitesmoke}{rgb}{0.96, 0.96, 0.96}
\newcommand{\tuple}[1]{\ensuremath{\left\langle #1 \right\rangle}}
\def\<{\langle}
\def\>{\rangle}
\def\jg#1{\leavevmode\llap{#1}}
\def\bad{\leavevmode\llap{*}}
\newcommand{\mdef}{\buildrel \text{d{}ef}\over =}
\newcommand{\light}[1]{\textcolor{grey}{#1}}

\newcommand*{\defeq}{\mathrel{\vcenter{\baselineskip0.5ex \lineskiplimit0pt
                     \hbox{\scriptsize.}\hbox{\scriptsize.}}}%
                     =}
%\renewcommand*{\bibfont}{\scriptsize}

\setbeamerfont{bibliography item}{size=\tiny}
\setbeamerfont{bibliography entry author}{size=\tiny}
\setbeamerfont{bibliography entry title}{size=\tiny}
\setbeamerfont{bibliography entry location}{size=\tiny}
\setbeamerfont{bibliography entry note}{size=\tiny}
\DeclareMathOperator*{\argmax}{arg\,max}
\DeclareMathOperator{\argmaxL}{arg\,max}


%%%%%%%%%%%%%%%%%%%%% Headling deletion%%%%%%%%%%%%%%%%%%%%%
\makeatletter
    \newenvironment{withoutheadline}{
        \setbeamertemplate{headline}[default]
        \def\beamer@entrycode{\vspace*{-\headheight}}
    }{}
\makeatother
%%%%%%%%%%%%%%%%%%%%%%%%%%%%%%%%%%%%%%%%%%%%%%%%%%%%%%%%%%%%
\beamertemplatenavigationsymbolsempty
%%%%%%%%%%%%%%%%% Block color%%%%%%%%%%%%%%%%%%%%%%%%%%%%%%
\definecolor{mygreen}{cmyk}{0, 1, 1, 0.5} %{0.82,0.11,1,0.25} 
\setbeamertemplate{blocks}[rounded][shadow=false]
\addtobeamertemplate{block begin}{\pgfsetfillopacity{0.8}}{\pgfsetfillopacity{1}}
\setbeamercolor{structure}{fg=mygreen}
\setbeamercolor*{block title example}{fg=blue!50,
bg= blue!10}
\setbeamercolor*{block body example}{fg= blue,
bg= blue!5}

%%%%%%%%%%%%%%%%%%%%%%%%%%%%%%%%%%%%%%%%%%%%%%%%%%%%%%%%%%

%%%%%%%%%%%%%%%%%%%% Numbering%%%%%%%%%%%%%%%%%%%
\resetcounteronoverlays{exx}
%%%%%%%%%%%%%%%%%%%%%%

%%%%%%%%%%%%%% Appendix %%%%%%%%%%%%%%%%%%%%%

%%%%%%%%%%%%%%%%%%%%%%%%%%%%%%%%%%%%%%

\begin{document}

\renewcommand{\inserttotalframenumber}{22}
\title{Learning biases may prevent lexicalization of pragmatic inferences}%\\[0,1cm] 
\subtitle{A case study combining iterated (Bayesian) learning\\ and functional selection}
%\subtitle{An overview}
\author{\center Thomas Brochhagen\inst{1} \and Michael Franke \inst{2} \and Robert van Rooij \inst{1}}
\institute{\inst{1} ILLC, University of Amsterdam \and \inst{2} Department of Linguistics, University of T\"ubingen\vspace{1cm}}

\date{\center \scriptsize CogSci 2016, Philadelphia, USA, 2016.08.13}

%%%%
\metroset{sectionpage=none}
\begin{frame}
\titlepage
\end{frame}
%
\begin{withoutheadline}

\section{Introduction}
\begin{frame}
  	\underline{Main methodological contribution}

	\begin{itemize}
        \item interaction of: \hfill (replicator mutator dynamics)
          \begin{itemize}
          \item fitness-relative replication \hfill (replicator dynamics)
          \item iterated learning \hfill (mutator dynamics)
          \end{itemize}
        \item learners perform joint-inference:
          \begin{itemize}
          \item type of pragmatic behavior \\
            \textcolor{gray}{RSA-style probabilistic types (Frank \& Goodman 2012)}
          \item lexical meaning \\
            \textcolor{gray}{LOT-style learning biases (Piantadosi et al. under review)}
          \end{itemize}

	\end{itemize}

        \medskip

        \underline{Case study on scalar implicatures}

        \begin{itemize}
        \item model shows prevalence of:
          \begin{itemize}
          \item Gricean pragmatic use
          \item non-lexicalized upper-bounds
          \end{itemize}
        \end{itemize}
\vFill
\begin{block}{}
	{\tiny \bibj{frank+goodman:2012} \bibj{piantadosi+etal:underreview}}
\end{block}

\end{frame}


\begin{frame}
	\frametitle{The semantics-pragmatics distinction}
	\alert{Semantics}\\ Literal meaning (truth-conditional)\\ \vspace{2cm}
	\alert{Pragmatics}\\ Information beyond literal meaning (e.g. defeasible inferences)

\end{frame}
\begin{frame}
	\frametitle{Scalar inferences}
%\vspace{1cm}
\begin{exe}
\itemsep1em
\item \tuple{\alert{some}, many, most, \alert{all}}\\ \begin{xlist} \itemsep1em
	\item All students came to class\\ $\rightarrow$ \alert{Some} students came to class
	\item Some students came to class\\ $\leadsto$ \alert{Not all} students came to class
	\end{xlist}
\item \tuple{may, should, must}
\item \tuple{one,two,three, ...}
\item \tuple{or, and}
\item ...
\end{exe}

\end{frame}

\begin{frame}
  \begin{center}
	  The use of a less informative expression when a more informative one \underline{could have been used}^{\bf{\alert{*}}} can license a defeasible inference that stronger alternatives do not hold
  \end{center}

\vspace{3cm}

\noindent {\bf \alert{*}}The hearer assumes the speaker to be knowledgeable and cooperative

\vFill

\begin{block}{}
	{\tiny \bibj{horn:1972} \bibj{gazdar:1979} \bibj{grice:1975}}
\end{block}
\end{frame}


\begin{frame}

  \begin{enumerate}
	  \item Why are (pragmatically inferred) upper-bounds of weak(er) alternatives not part of semantics? \label{des-1}
	  \item What justifies semantic structure in light of pragmatic enrichment? \label{des-2}
  \end{enumerate}
  
%  \vspace{2cm} 
%Today's talk:
%  \begin{itemize}
%	  \item Propose a model to address (\ref{des-2}) and analyze dynamics of linguistic pressures more broadly
%	  \item Use (\ref{des-1}) as a case study for (\ref{des-2}) 
%  \end{itemize}
\end{frame}


%\begin{frame}
%  \frametitle{Explanations for a lack of semantic upper-bounds I}
% \begin{center}
%  (1) Signal no commitment to stronger alternatives when knowledge/cooperativity are not given
%  \end{center}
%
%%Cf. lexicalizing `some' and `sbna'
%
%Conditions that may pressure for English-like semantics:
%\begin{itemize}
%  \item Contextual cues are very reliable
%  \item Morphosyntactic disambiguation is not frequently necessary
%  \item Morphosyntactic disambiguation is not very costly 
%  \item Cost of larger lexica higher than morphosyntactic disambiguation
%\end{itemize}
%\end{frame}
%
%\begin{frame}
%  \frametitle{Explanations for a lack of semantic upper-bounds II}
%\begin{center}
%    (2) Lack of upper-bounds provides learnability advantage
%\end{center}
%
%\end{frame}

\metroset{sectionpage=progressbar}


\section{Model}
\begin{frame}
	\frametitle{Components I: Probabilistic (pragmatic) language users}
\begin{center}
\includegraphics[scale=0.27]{player-type}
\end{center}
%	{\bf Relation that is easier to visualize, e.g. a box enclosing language, then another surrounding it with signaling behavior, and then label that as a type}
%	\vspace{1cm}
%\begin{itemize}\itemsep2em
%	\item Varied lexica 
%	\item Varied production and comprehension behavior \makebox(0,0){\put(0,5\normalbaselineskip){%
%               $\left.\rule{0pt}{3.5\normalbaselineskip}\right\}$ A player's type}} 
%\end{itemize}

\vFill
\begin{block}{}
	{\tiny \bibj{benz+etal:2005} \bibj{bergen+etal:2016} \bibj{frank+goodman:2012} \bibj{franke+jaeger:2014}}
\end{block}
\end{frame}

\begin{frame}
	\frametitle{Lexica}
	\begin{itemize}
		\item $s_1$: Bill read some but not all books
		\item $s_2$: Bill read all books
	\end{itemize}
\vspace{1cm}
\begin{centering}
	 $L_{\text{lack}}$ = \bordermatrix{~ & m_{\text{all}} & m_{\text{some}} \cr 
                  s_1 & 0 & 1 \cr
s_2 & 1 & 1 \cr}\\[1.5cm]
\end{centering}

\begin{centering}
	$L_{\text{bound}}$ = \bordermatrix{~ & m_{\text{all}} & m_{\text{some}} \cr 
	s_1 & 0 & 1 \cr
s_2 & 1 & 0 \cr}\\
\end{centering}

\end{frame}


\begin{frame}
	\frametitle{Literal behavior} 	

\includegraphics[width=\linewidth,height=\textheight,keepaspectratio]{literal}
\end{frame}


\begin{frame}
\frametitle{Pragmatic behavior}
\includegraphics[width=\linewidth,height=\textheight,keepaspectratio]{griceanb}

\end{frame}

\begin{frame}
	\frametitle{Components II: Cultural transmission}

Two competing pressures:
\vspace{0,5cm}
\begin{enumerate} \itemsep1em
	\item Communicative efficiency\\[0,3cm] ... as replicator dynamics; $\dot{x}$ 
	\item Learnability \\[0,3cm] \makebox(0,0){\put(155,8.9\normalbaselineskip){%
               $\left.\rule{0pt}{4.5\normalbaselineskip}\right\}$ Replicator-mutator dynamics}}
	... iterated Bayesian learning \hspace{2.5cm} $\hat{x} =  \dot{x} \cdot Q$\\ 
	\hspace{0.47cm} as mutator dynamics; $Q$
		\end{enumerate}
\vFill
\begin{block}{}
	{\tiny \bibj{griffiths+kalish:2007} \bibj{nowak+krakauer:1999}}
\end{block}
\end{frame}


\begin{frame} 
	\frametitle{Functional pressure (replicator dynamics); $\dot{x}_i = \frac{x_i f_i}{\Phi}$
}

	\begin{itemize}\itemsep1.5em
			\item Population of types $x$\\[0,3cm] $x_i$ is the proportion of $t_i$ in $x$
			\item Fitness of type $i$\\[0,3cm] $f_i =  \sum_j x_j U(x_i,x_j)$
			\item Average fitness in the population\\[0,3cm] $\Phi = \sum_i x_i f_i$
	\end{itemize}
\end{frame}


%\begin{frame}
%	\frametitle{I. Iterated learning (mutator dynamics): Parent data}
%\includegraphics[width=\linewidth,height=\textheight,keepaspectratio]{parent}
%
%\end{frame}

\begin{frame}
	\frametitle{Iterated learning (mutator dynamics)}

\includegraphics[width=\linewidth,height=\textheight,keepaspectratio]{learning}
\end{frame}



\section{Analysis}
\begin{frame}
	\frametitle{Lexica, signaling behavior \& types}
\underline{Lexica subset}
\begin{table}
\centering 
\begin{tabular}{l c l}
$L_{\text{tautology}}$ = $\begin{pmatrix} 1 & 1 \\ 1 & 1 \end{pmatrix}$ &
$L_{\text{bound}}$ = $\begin{pmatrix} 0 & 1 \\ 1 & 0 \end{pmatrix}$ &
$L_{\text{lack}}$ = $\begin{pmatrix} 0 & 1 \\ 1 & 1 \end{pmatrix}$
\end{tabular}
\end{table}

%\begin{table}
%\centering 
%\begin{tabular}{l c l}
%$L_1$ = $\begin{pmatrix} 0 & 0 \\ 1 & 1 \end{pmatrix}$ & 
%$L_2$ = $\begin{pmatrix} 1 & 1 \\ 0 & 0 \end{pmatrix}$ & 
%
%$L_4$ = $\begin{pmatrix} 0 & 1 \\ 1 & 0 \end{pmatrix}$ &
%$L_5$ = $\begin{pmatrix} 0 & 1 \\ 1 & 1 \end{pmatrix}$ &
%$L_6$ = $\begin{pmatrix} 1 & 1 \\ 1 & 0 \end{pmatrix}$
%\end{tabular}
%\end{table}

\underline{Signaling behavior}\\ {\em Literal} or {\em pragmatic}\\[0.75cm]
\underline{Types}\\ $12$ types ($2$ behaviors $\times$ $6$ lexica)
\end{frame}

\begin{frame} 
\begin{center}What factors lead to the selection of $L_{\text{lack}}$-like semantics?\end{center}
\bigskip
\begin{table}
\centering
\begin{tabular}{l|l|l}
    parameter & explanation & locus\\ \hline
    $\alert{c} \in [0,1]$ & learning bias for upper-bound lack &  $P(t_i)$\\
    $\alert{l} \geq 1$ & sampling to MAP & $[P(t_i)P(d|t_i)]^l$\\
    \textcolor{gray}{$\lambda \geq 1$} & \textcolor{gray}{rationality parameter} & \textcolor{gray}{$\text{exp}(\lambda \; R_{n-1}(s|m;L))$}\\
    \textcolor{gray}{$k = |d|$} & \textcolor{gray}{datum length} & \textcolor{gray}{$P(d|t_j)P(t_i|d)$}\\
    \textcolor{gray}{$|D|$} & \textcolor{gray}{data produced per parent type} & \textcolor{gray}{$P(d|t_j)P(t_i|d)$}
\end{tabular}
\end{table}
\end{frame}

\begin{frame}
\frametitle{Expressivity only}
\includegraphics[width=\linewidth,height=\textheight,keepaspectratio]{cogsci-fitnessonly}

\end{frame}

\begin{frame}
\frametitle{Learnability only}
\includegraphics[width=\linewidth,height=\textheight,keepaspectratio]{cogsci-learningonly}

\end{frame}

\begin{frame}
	\frametitle{Expressivity and learnability}

\includegraphics[width=\linewidth,height=\textheight,keepaspectratio]{cogsci-fitnessandlearning-l1}

\end{frame}

%\begin{frame}
%  \frametitle{Interim remarks}
%  \begin{itemize}
%	\item Lack of semantic-upper bounds can overcome pressures and stabilize in a population provided...
%	  \begin{itemize}
%		\item Bias for simpler representations
%		\item Pragmatics to compensate lack of upper-bounds in use
%	  \end{itemize}
%%	\item Weak learning bias sufficient for selection of $L_5$ over $L_5$ (and the latter's incumbency)
%  \end{itemize}
%  But
%
%  \begin{itemize}
%    \item Highly polymorphic populations even for high $c$
%    \item Role of parameters unexplored
%  \end{itemize}
%
%\end{frame}

\begin{frame}
	\frametitle{Effect of prior with higher posterior maximization}
\includegraphics[width=\linewidth,height=\textheight,keepaspectratio]{cogsci-fitnessandlearning-l3}

\end{frame}

\begin{frame}
	\frametitle{Prior and posterior}

\includegraphics[width=\linewidth,height=\textheight,keepaspectratio]{cogsci-heatmap}

\end{frame}

\metroset{sectionpage=none}


\section{Conclusion \& Outlook}
\begin{frame}
  \frametitle{Concluding remarks: Application}
  \begin{itemize}
		\item Learnability steers language\\ towards simpler semantics
		\item Pragmatics compensates for \makebox(0,0){\put(0,4\normalbaselineskip){%
               $\left.\rule{0pt}{2.5\normalbaselineskip}\right\}$ Lack of semantic upper-bounds}} \\
 potential loss in expressivity  
%		\item Functional advantages 
%		\begin{itemize}
%		  \item Reliability of contextual cues to cancel implicatures
%		  \item Lexicon size
%		  \item ...
%		\end{itemize}
 \end{itemize}
\vspace{1.5cm}
Provided
\begin{itemize} 
  \item Some degree of rationality in learning \& choice
\end{itemize}
\end{frame}

%\begin{frame}
%	\frametitle{Concluding remarks: Model}
%Combines
%\begin{itemize}\itemsep1em
%		    \item Functional pressure
%		    \item Learning pressure 
%		    \item (Probabilistic) hearer \& speaker models
%		    \item Distinct languages
%	\end{itemize}
%\end{frame}

%\begin{frame}
%  \frametitle{Future directions \& details}
%Not covered
%\begin{itemize}\itemsep0.75em
%    \item Effects of variation in learning data (size and datum length)
%    \item Interaction of different parameters (e.g. $\lambda$ and $l$)
%    \item ...
%\end{itemize}
%\vspace{1cm}
%Future
%    \begin{itemize}\itemsep0.75em
%      \item Frequency effects
%      \item Larger lexica \& other functional (dis)advantages 
%      \item Further applications
%    \end{itemize}
%\end{frame}

\begin{frame}

  \frametitle{Selection, learning, pragmatic use \& lexical meaning}

  	\underline{Main methodological contribution}

	\begin{itemize}
        \item interaction of: \hfill (replicator mutator dynamics)
          \begin{itemize}
          \item fitness-relative replication \hfill (replicator dynamics)
          \item iterated learning \hfill (mutator dynamics)
          \end{itemize}
        \item learners perform joint-inference:
          \begin{itemize}
          \item type of pragmatic behavior
          \item lexical meaning
          \end{itemize}

	\end{itemize}

        \medskip

        \underline{Case study on scalar implicatures}

        \begin{itemize}
        \item model shows prevalence of:
          \begin{itemize}
          \item pragmatic use
          \item non-lexicalized upper-bounds
          \end{itemize}
        \end{itemize}
\end{frame}


%%%%%%%%%%%%%%%%%%%% BIBLIOGRAPHY%%%%%%%%%%%%%%%%%%%%%%%%%%%%%%%%%%%%%%%
\section[References]{References}
\begin{frame}[allowframebreaks]\frametitle{References}
\tiny
\setbeamertemplate{bibliography item}[text]
%\begin{multicols}{2}
\bibliographystyle{alpha} %
\bibliography{./../paper/bounds-rmd}
%\end{multicols}
\end{frame}
\end{withoutheadline}
%%%%%%%%%%%%%%%%%%%%%%%%%%%%%%%%%%%%%%%%%%%%%%%
\end{document}

