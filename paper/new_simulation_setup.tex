\documentclass[a4paper]{article}

\usepackage{geometry}
\usepackage{natbib}
\bibpunct[:]{(}{)}{,}{a}{}{;}

%--------------------
%\usepackage{gb4e}
%\noautomath

\usepackage{amsmath}
\usepackage{amsfonts}
\usepackage{amsthm}
\usepackage{amssymb}
\usepackage{mathrsfs}
\usepackage{nicefrac}
%\usepackage{stmaryrd}
%\usepackage{multicol}
\usepackage{graphicx}
\usepackage{color}
\usepackage{booktabs}
%\newcommand{\mvalueof}[1]{\llbracket#1\rrbracket}
\newcommand{\citeposs}[2][]{\citeauthor{#2}'s (\citeyear[#1]{#2})}
\newcommand{\tuple}[1]{\ensuremath{\left\langle #1 \right\rangle}} 

\newcommand{\hl}[1]{\textcolor[rgb]{.8,.33,.0}{#1}}% prints in orange
%\newcommand{\argmax}[1]{\underset{#1}{\operatorname{arg}\,\operatorname{max}}\;}
%\newcommand{\argmin}[1]{\underset{#1}{\operatorname{arg}\,\operatorname{min}}\;}
%\newcommand{\sbna}{\exists\lnot\forall}

\definecolor{Red}{RGB}{178,34,34}
\newcommand{\mf}[1]{\textcolor{Red}{[MF: #1]}} 
\newcommand{\tb}[1]{\textcolor[rgb]{.8,.33,.0}{[TB: #1]}}% prints in orange

\usepackage{blkarray}
\usepackage{xspace}

%%% MF's commands
\newcommand{\set}[1]{\left\{#1\right\}}
\newcommand{\card}[1]{\left \lvert \, #1 \, \right\rvert}
\newcommand{\abs}[1]{\lvert #1 \rvert}
\newcommand{\States}{\ensuremath{S}\xspace}		% Set of States
\newcommand{\state}{\ensuremath{s}\xspace}		% single states
\newcommand{\mystate}[1]{\ensuremath{\state_{\text{#1}}}\xspace} %meaningful states
\newcommand{\Messgs}{\ensuremath{M}\xspace}		% Set of Messages
\newcommand{\messg}{\ensuremath{m}\xspace}		% single messages
\newcommand{\mymessg}[1]{\ensuremath{\messg_{\text{#1}}}\xspace} %meaningful messages
\newcommand{\ssome}{\mystate{\ensuremath{\exists\neg\forall}}}
\newcommand{\sall}{\mystate{\ensuremath{\forall}}}
\newcommand{\snone}{\mystate{\ensuremath{\emptyset}}}
\newcommand{\msome}{\mymessg{some}}
\newcommand{\mall}{\mymessg{all}}
\newcommand{\asome}{\myact{\ensuremath{\exists\neg\forall}}}
\newcommand{\aall}{\myact{\ensuremath{\forall}}}
\definecolor{mygray}{cmyk}{0.35,0.35,0.35,0.35}
\newcommand{\mygray}[1]{{\textcolor{mygray}{#1}}}
%%% 


%--------------------
%
%\usepackage{setspace}
%\onehalfspacing
%
%-------------------


\title{Tracing the cultural evolution of meaning at the semantics-pragmatics interface}

\author{%\bf NAME1 and NAME2\\
    ( -- draft \today --- )
}


\date{}

\begin{document}

We consider three states $\States = \set{\snone, \ssome, \sall}$. This state space can be
thought of as a partition of possible worlds into cells where none, some or all of the $A$s are
$B$s, for some arbitrary fixed predicates $A$ and $B$. Eight concepts can be distinguished
based on their truth or falsity in three world states, six of which are not contradictory
(always false) or tautologous (always true). These are listed with mnemonic names in
Table~\ref{tab:concepts}. 

\begin{table}
  \centering
\begin{center}
  \begin{tabular}{lccclc}
    \toprule
    intuitive name
    & \snone
    & \ssome
    & \sall
    & least complex formula
    & complexity
    \\ \midrule
    ``all''
    & 0
    & 0
    & 1
    & $A \subseteq B$
    & $3$
    \\
    ``some but not all''
    & 0
    & 1
    & 0
    & $A \cap B \neq \emptyset \wedge A \neq \emptyset$
    & $8$
    \\    
    ``some''
    & 0
    & 1
    & 1
    & $A \cap B \neq \emptyset$
    & $4$
    \\
    ``none''
    & 1
    & 0
    & 0
    & $A \cap B = \emptyset$
    & $4$
    \\
    ``none or all''
    & 1
    & 0
    & 1
    & $\neg(A \cap B \neq \emptyset \wedge A \neq \emptyset)$
    & $9$
    \\
    ``not all''
    & 1
    & 1
    & 0
    & $\neg (A \subseteq B)$
    & $5$
    \\
    \bottomrule
  \end{tabular}
\end{center}
\caption{Available concepts and their minimal derivation length}
\label{tab:concepts}
\end{table}

A lexicon $L$ is a mapping $\Messgs \rightarrow C$ from messages to concepts. With three
messages there are $6^3 = 343$ possible lexica, but many of these assign the same concept to
more than one message. For simplicity, we restrict attention to only those lexica which avoid
expressive redundancy. Hardcoding such an \emph{mutual exclusivity bias}
\citep[e.g.][]{Clark2009:Lexical-Meaning}, there are 20 non-redundant lexica. Functional
pressure towards efficient communication will evidently favor non-redundant lexica, so this
restriction is fairly innocuous but practical.

As before, the type of a player is a combination of a lexicon and a manner of pragmatic
language use: literal or pragmatic. With this, there are 40 types in this model.

\begin{table}
  \centering
  \begin{align*}
    C & \rightarrow_2 C \wedge C 
    & 
    X & \rightarrow_1 \set{A,B} \\
    C & \rightarrow_2 \neg C 
    & 
    X & \rightarrow_1 X \cap X \\
    C & \rightarrow_1 X \subseteq X
    & 
    X & \rightarrow_1 X \cup X \\
    C & \rightarrow_1 X \neq \emptyset \\
    C & \rightarrow_1 X = \emptyset     
  \end{align*}
  \caption{Toy grammar in a set-theoretic "language of thought"}
  \label{tab:grammar}
\end{table}


The prior probability of a type is just the prior probability of its lexicon. The prior of a
lexicon is a function of the complexity of the concepts in its image set. Lexica that use
simpler concepts are \emph{a priori} more likely. This can be motivated by assuming that
learners beam search for suitable concepts to map onto overt signals by (probabilistically)
considering simpler concepts first. Many ways for defining complexity of a concept are
conceivable. If strong empirical claims were at stake here, empirically motivated measures of
complexity should be used. For the sake of a non-trivial example, we follow
\citet{piantadosi+etal:underreview} and related work to define complexity of a concept as a
function of its derivation cost in a (weighed or probabilistic) generative ``language of
thought''. For concreteness of example, consider the toy grammar of concepts in
Table~\ref{tab:grammar}. This grammar uses basic set-theoretic operations to form expressions
which can be evaluated as true or false in our three world states. Applications of generative
rules have a cost attached to them. (Alternatively, a probability.) Here we simply assume that
Boolean combinations of concepts are more complex than ``atomic'' concepts and that otherwise
each rule application adds the same cost unit. Table~\ref{tab:concepts} lists, for each
concept, the least complex formula derivable in this grammar that has the appropriate truth
conditions. A simple way of defining priors over a lexicon is:
\begin{align*}
  P(L)  & = \prod_{c \in Im(L)} P(c)  \ \ \ \text{, with} & 
  P(c) & \propto \max_{c'}Compl(c') - Compl(c') + 1\,,
\end{align*}
where $Compl(c)$ is the complexity of each concepts.

\bibliographystyle{plainnat}
\bibliography{./bounds-rmd}
\end{document}
